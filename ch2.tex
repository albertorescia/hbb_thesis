\documentclass[10pt,a4paper]{book}
\usepackage[utf8]{inputenc}
\usepackage[english]{babel}
\usepackage{amsmath}
\usepackage{mathtools}
\usepackage{array}
\usepackage{booktabs}
\usepackage{gensymb}
\usepackage{slashed}
\usepackage{physics}
\usepackage{bbold}
\usepackage{stackengine}
\usepackage{amsfonts}
\usepackage{amssymb}
\usepackage{graphicx}
\usepackage{geometry}
\usepackage{pdfpages}
\usepackage{hyperref}
\usepackage{subcaption}
\usepackage[numbers,sort&compress]{natbib}


\newtheorem{theorem}{Theorem}[section]
\newcolumntype{L}{>{$}c<{$}}
\newcommand\todo[1]{\textcolor{red}{#1}}

\begin{document}
One of the best ways to test the Standard Model is through high-energy particle collisions. The LHC is a 27 km hadron-hadron circular collider where protons or nucleons interact with each other in a multitude of ways, resulting in a myriad of possible final states involving particles which do not exist in our cold universe. By identifying these final states, and selecting those which correspond to processes of interest, it is possible to study these processes in detail. A number of detectors, including ATLAS and CMS, lie on the beam pipe for this purpose.

The LHC has undergone several phases. During Run 1 (2009-2013), the LHC ran at a center of mass energy $\sqrt{s}$ of 7-8 TeV. The energy was increased to $\sqrt{s} = 13$ TeV during Run 2 (2015-2018). 
Run 2 delivered a total integrated luminosity of 156 fb$^{-1}$. During Run 3, set to begin next year, the center of mass energy could increase up to 14 TeV, and it is expected that the integrated luminosity will double to 300 fb$^{-1}$. 

In this chapter, we will shortly describe the physics of hadron colliders.

\section{Factorization}

%Walk through a collision in detail, CM energy, set-up of experimental variables, ecc.
  
The proton is a dynamic system. In a simplistic view, it is composed of three valence quarks, $u, u, d$ bound together by gluons. The gluons interact both with the valence quarks and themselves, leading to a ``sea'' composed of gluons, quarks and anti-quarks of all flavors, which originate from gluon splitting. The sea is dominant at low energies, and suppressed at higher energies. 

When two hadrons collide at high energies, the resulting interaction does not directly involve the hadrons as a whole but the \emph{partons} which constitute the hadron. Thanks to asymptotic freedom, these partons are quasi-free. This means that, even when the interaction with the hadron as a whole is deeply inelastic, the parton-parton interaction which occurs is instead elastic. 

Relativistic considerations also allow us to deduce that the time-scale is such that only interactions with one parton per hadron are possible. Indeed, in the rest frame of the proton, the time-scale of the interactions holding the proton together are of the order $1/m_p$. In the laboratory frame of the collision, this is boosted by a factor $\gamma = \sqrt{s}/2m_p$. Since the energies available at the LHC place us firmly in the ultrarelativistic limit, interactions with a virtual particle of energy $Q^2 \gg m_p^2$ occur on a time-scale much shorter than $\gamma/m_p$, the parton probed has no time to communicate with the other partons.

In the case of deep inelastic scattering, after the constituent parton has been struck, the virtual particles emitted by the constituent as part of normal interactions within the hadron can no longer be reabsorbed. This effect is exacerbated at higher $Q$. The end result is a perturbative evolution of the final state particles from the interaction, together with these liberated virtual particles, down to energies of the order of the Landau pole of QCD, $\Lambda_{QCD}$. At these energies, due to the running of the coupling constant $\alpha_S$, it is no longer possible to describe QCD using perturbative physics. What follows is the hadronization process, where final state particles undergo non-perturbative interactions which transform them into relatively long-lived hadrons. Since the time-scale for hadronization is much longer when compared to the elementary process, the cross section for the collision is said to be \emph{factorized} into a hard process described by an elementary cross section and functions describing the non-perturbative physics involved in the hadron before the interaction as well as the hadronization process. 


\begin{figure}[h]
\centering
\includegraphics[scale=1.0]{ch2_images/factorization}
\caption{A schematic representation of factorization in a process which results in the production of pion in a proton-proton collision \cite{Soffer}.}
\label{factorization pion}
\end{figure}

This discussion can be neatly summarized in a formula:
\begin{equation}
\frac{d\sigma}{dX} = \sum_{j,k} \int dx_1 dx_2 \; f_j(x_1, Q^2)f_k(x_2, Q^2) \; \frac{d\hat{\sigma}_{jk}(x_1P_1, x_2P_2, Q^2, \mu_F^2)}{d\hat{X}} \; F(\hat{X}\rightarrow X; Q^2, \mu_F^2).
\label{master formula}
\end{equation}
Here, we are stating that the differential cross section with respect to a hadronic observable $X$ can be written in terms of the parton-level cross section $d\hat{\sigma}/d\hat{X}$, differential in the parton-level observable $\hat{X}$. $f_j(x_1P_1, \mu_F)$ and $f_k(x_2P_2, Q^2)$ are known as Parton Distribution Functions (PDFs) and, at the lowest order in perturbation theory, describe the probability of extracting a parton of type $j$ or $k$ with momentum fraction $x_1$ or $x_2$, respectively from the colliding hadrons with momenta $P_1$ and $P_2$ when probed at energy $Q^2$. We must sum over all possible partons, and integrate over all momentum fractions. Finally, the function $F(\hat{X}\rightarrow X; Q^2, \mu_F^2)$ describes the (non-perturbative) transition from partonic states to hadronic states and indicates how to relate any partonic observable $\hat{X}$ with the measured hadronic observable $X$. This can be, for example, a Fragmentation Function, used when a single final-state hadron is observed, or a Jet Function, used instead when describing an aggregate transition to hadrons. We will describe all of the involved functions in more detail in subsequent sections. 

Figure \ref{factorization pion} is a schematic representation of factorization in a proton-proton collision described by Equation (\ref{master formula}).

\section{Parton Distribution Functions}

As previously stated, Parton Distribution Functions contain information regarding the constituents of a hadron. To better understand them, we shall study in detail a benchmark process, known as Deep Inelastic Scattering (DIS).

\subsection{Deep Inelastic Scattering}

DIS is a process which investigates the insides of hadrons using leptonic probes. In this section we will focus on high-energy electron-proton inelastic scattering. In this process, the incoming electron exchanges a virtual photon (if below the production threshold of the massive gauge bosons) with the proton. Specifically, we will look at the process
\begin{equation}
e^- \; p \rightarrow e^- \; X
\end{equation}
where X represents an undetermined hadronic final state, mediated by a virtual photon. 

\subsubsection{Kinematics} 
A number of kinematic variables are needed to fully describe the process. Without loss of generality, we will work in the target rest frame (TRF), where the struck proton is at rest. With reference to Figure \ref{DIS diagram}, we have
\begin{equation}
\begin{cases}
k^\mu = (E, 0, 0, E)  \\
p^\mu = (m_p, 0, 0, 0) \\
k^{\mu\prime} = (E^\prime, E^\prime \sin\theta, 0, E^\prime \cos\theta) \\
q^\mu = k^\mu - k^{\mu\prime}
\end{cases}
\label{kinematics}
\end{equation}
where $k^\mu$ and $k^{\mu\prime}$ refer to the four-momenta of the initial and final state electron, respectively, $p^\mu$ the four-momentum of the initial state proton and $q^\mu$ to the four momentum of the virtual photon. The quantities $E$ and $E^\prime$ refer to the respective energies of the electron in the initial and final state, and $\theta$ is the angle at which the electron is scattered with respect to its initial momentum $\vec{k}$. 

\begin{figure}
\centering
\includegraphics[scale=0.25]{ch2_images/dis}
\caption{A schematic representation of the DIS process.}
\label{DIS diagram}
\end{figure}

The first kinematic variable to consider is the energy of the virtual probe, known also as the scale of the process. Since DIS is a space-like process, the four-momentum of the virtual probe $q^2 < 0$, therefore we take
\begin{equation}
Q^2 = -q^2.
\end{equation}
Next we introduce a series of Lorentz-invariant variables. The first variable, $y$, is defined as
\begin{equation}
y = \frac{p\cdot q}{p\cdot k} \overset{\mathrm{TRF}}{=} 1 - \frac{E^\prime}{E}.
\end{equation}
As is clear, in the TRF, $y$ represents the fraction of energy lost during the inelastic process. For this reason, $0 \leq y \leq 1$.
Next, we introduce a similar variable
\begin{equation}
\nu = \frac{p\cdot q}{m_p} \overset{\mathrm{TRF}}{=} E - E^\prime
\end{equation}
which in the TRF represents the the energy lost by the electron in the scattering process.
Finally, we introduce the variable Bjorken $x$, defined as
\begin{equation}
x = \frac{Q^2}{2p\cdot q} \overset{\mathrm{TRF}}{=} \frac{Q^2}{2m_p\nu}
\end{equation} 
If we consider the definition of the invariant mass of the system,
\begin{equation}
W^2 = (p + q)^2 = p^2 + 2p\cdot q + q^2 = m_p^2 + Q^2\left(\frac{1}{x} - 1\right) \geq m_p^2
\end{equation}
it becomes clear that $x$ represents the ``elasticity'' of the process: $x$ is limited to the range $0 \leq x \leq 1$, and $x = 1$ corresponds to a perfectly elastic collision, whereas $x = 0$ corresponds to a perfectly inelastic one. By definition, the Deep Inelastic limit of this process is the limit in which $Q^2 \rightarrow \infty$ while $x$ is held constant.
 
\subsubsection{Cross Section}

It is possible to calculate the cross section for the inelastic scattering process described above. The most general Lorentz-invariant cross section for the interaction considered is
\begin{equation}
\frac{d^2\sigma}{dxdQ^2} = \frac{4\pi\alpha^2}{Q^4}\left[\left(1 - y - \frac{m_p^2 y^2}{Q^2} \right)\frac{F^2(x,Q^2)}{x} + y^2F_1(x,Q^2)\right].
\end{equation}
In the deep inelastic limit, this simplifies to
\begin{equation}
\frac{d^2\sigma}{dxdQ^2} \approx \frac{4\pi\alpha^2}{Q^4}\left[\left( 1 - y \right) \frac{F^2(x,Q^2)}{x} + y^2F_1(x,Q^2)\right]. 
\label{Rutherford}
\end{equation}
$F_1$ and $F_2$ are \emph{structure functions}, which describe the internal structure of the proton.

\subsubsection{Bjorken Scaling and the Callan-Gross Relation}
An important observation involving the aforementioned structure functions is \emph{Bjorken scaling}. This is the observation that, to first order, $F_1$ and $F_2$ are independent of $Q^2$, as shown in Figure \ref{Bjorken scaling}.
\begin{figure}[hp]
\centering
\includegraphics[width=\textwidth]{ch2_images/bjorken_scacling}
\caption{The observation of approximate scaling of the structure functions at various experiments. The observed violations are caused by higher-order contributions \cite{pdg}.}
\label{Bjorken scaling}
\end{figure}
In addition to this scaling, experimental observations also found that, at sufficiently large $Q^2$
\begin{equation}
F_2(x) = 2xF_1(x).
\end{equation}
This relation is known as the \emph{Callan-Gross relation} and is shown in Figure \ref{Callan gross}.
\begin{figure}[ht]
\centering
\includegraphics[scale=0.5]{ch2_images/callan}
\caption{An experimental observation of the Callan-Gross relation \cite{Griffiths:2008zz}.}
\label{Callan gross}
\end{figure}
We would expect to see scaling if the scattering occurred against point-like particles, giving evidence to the composite nature of the proton. In addition to this, the Callan-Gross relation tells us that the constituent partons carry spin-1/2. We therefore have experimental evidence to support the statement that the proton is composed of point-like spin-1/2 particles, namely quarks\footnote{The proton also has a significant gluon component, as will be discussed in a later section.}. The incoming electron elastically scatters against these constituents, explaining these observations and justifying the formula (\ref{master formula}).

If we consider the DIS process in the infinite momentum frame, defined as the frame in which the energy of the proton in the initial state $E_p \gg m_p$, i.e. $p^\mu = (E_p, 0, 0, E_p)$, we can deduce that the four-momentum of the struck quark can be written as
\begin{equation}
p^\mu_q = (\xi E_p, 0, 0, \xi E_p)
\end{equation}
where $\xi$ represents the fraction of the proton's momentum carried by the quark. The quark in the final state of the electron-quark scattering will have four-momentum
\begin{equation}
\left(\xi p + q \right)^2 = \xi^2 p^2 + 2\xi p \cdot q + q^2 = m_q^2.
\end{equation}
For this relation to hold, we must have $2\xi p \cdot q + q^2 = 0$, which implies that
\begin{equation}
\xi = \frac{-q^2}{2 p \cdot q} = \frac{Q^2}{2 p \cdot q} = x.
\end{equation}
We can therefore identify $x$ Bjorken with the momentum fraction of the struck constituent quark.

\subsubsection{Determining PDFs}
We can now proceed in actually determining the constituents of the proton. Experimentally, this is done using formula (\ref{master formula}), which in the case of DIS becomes
\begin{equation}
\frac{d\sigma}{dE^\prime d\Omega} = \sum_f \int_0^1 dx \; \frac{d\hat{\sigma}}{dE^\prime d\Omega}(xP, q) \; \phi_f(x).
\label{master formula DIS}
\end{equation}
where $\phi_f(x)$ represents the PDF for the (anti)quark of flavor $f$ within the proton. The partonic cross section is easily calculable in QED
\begin{equation}
\frac{d\hat{\sigma}}{dE^\prime d\Omega} = \frac{4\alpha^2}{Q^4}E^{\prime 2}\cos^2 \frac{\theta}{2}e_f^2 \frac{2m_fx}{Q^2}\delta \left(x^\prime - x\right) \left[1 + \frac{Q^2}{2m_f^2}\tan^2\frac{\theta}{2}  \right].
\label{partonic cross section}
\end{equation}
By inserting (\ref{Rutherford}) and (\ref{partonic cross section}) in (\ref{master formula DIS}), we find that
\begin{equation}
\begin{cases}
F_1 (x) = \frac{1}{2} \sum_f e_f^2 \phi_f (x) \\
F_2 (x) = x \sum_f e_f^2 \phi_f (x).
\label{Structure functions PDFs}
\end{cases}
\end{equation}
We have managed to describe the structure functions in terms of the PDFs, and in turn found the Callan-Gross relation!

By experimentally measuring the cross-section (\ref{master formula DIS}), we can deconvolve the PDF contribution. Figure \ref{PDF plot} shows the PDFs of various partons within the proton at $Q^2 = 10$ GeV$^2$. 

PDFs have the notable property of being \emph{universal}. This means that, regardless of what is used to probe them, they will always be the same since they are an intrinsic property of a given hadron. They are intrinsically non-perturbative, but their evolution with energy is governed by the renormalization group equation obtained in perturbation theory, as described in the below. 

Determining the proton PDFs is fundamental for making accurate predictions at the LHC, as they allow us to to obtain theoretical predictions for inclusive or exclusive variables, as in \ref{master formula}, and compare them to measured rates. 
\begin{figure}[ht]
\centering
\includegraphics[scale=0.4]{ch2_images/pdfs}
\caption{PDFs of the proton measured at various values of $x$ at $Q^2 = 10$ GeV$^2$ \cite{Metcalfe}.}
\label{PDF plot}
\end{figure}

\subsubsection{Gluon PDF}

The discussion up to now has focused exclusively on quarks, since, within the original framework of the so-called \emph{Quark Parton Model}, did not include gluons. However, Figure \ref{PDF plot} shows that, particularly at small $x$, i.e.\ at high energies, a significant fraction of the proton's constituents are gluons.

The gluon PDF can be determined in a manner analogous to the quark PDFs. With DIS, we must, however, consider higher-order QCD corrections to be able to study elementary processes such as $\gamma g \rightarrow q \overline{q}$. This is not ideal since the process of interest is suppressed. Alternatively, we can look at hadron-hadron collisions, where gluon interactions do appear at leading order for some processes.

\subsubsection{PDF Evolution}

In general, PDFs are functions both of the momentum fraction $x$ and the scale of the hard process $Q^2$. At first order, this dependence is negligible, but the inclusion of the gluon leads to higher-order QCD corrections which lead to the scaling violations observed. 

PDFs cannot be calculated from first principles; they must be measured from experiment. Thankfully, it is not necessary to perform measurements at different values of $x$ and $Q^2$ to fully determine PDFs.

The Dokshitzer-Gribov-Lipatov-Altarelli-Parisi (DGLAP) Equations \cite{GRIBOV197178, Altarelli:1977zs} allow us to calculate the evolution of a PDF from the scale $Q^2$ to a different scale $Q^{\prime 2}$. This is a notable achievement, as the PDFs can be measured at a certain energy, but re-utilized in a theoretical prediction for a process occurring at a different energy scale. 

For the sake of brevity, we shall limit ourselves to citing this incredible result, rather than deriving the whole equation.

The evolution equation for the quark parton density is:
\begin{equation}
\frac{dq_f(x,Q^2)}{d\log Q^2} = \frac{\alpha_s}{2\pi}\int_x^1 \, \frac{dy}{y} \left[q_f(y,Q^2)P_{qq}\left(\frac{x}{y} \right) + g(y,Q^2)P_{qg}\left(\frac{x}{y} \right) \right].
\end{equation}
This equation states that, to calculate the PDF for a parton of flavor $f$ at a given $x$ and $Q^2$ we need only to integrate the parton and gluon PDFs at $y$ and $Q^2$, along with the splitting functions $P_{qq}$ and $P_{qg}$. $P_{qq}$ gives the probability of finding a real quark $q$ with a certain momentum fraction after the emission of a virtual gluon, and $P_{qg}$ is the analogous function for the emission of a real quark after a gluon splitting.
\begin{equation}
\begin{cases}
P_{qq}(z) = C_F\frac{1 + z^2}{(1 - z)_+} \\
P_{qg}(z) = T_R\left[z^2 + (1-z)^2 \right]
\end{cases}
\end{equation}
where $C_F = (N_c^2 - 1)/2N_c$ is the Casimir invariant of $SU(3)$, $T_R$ is the trace of the Gell-Mann matrices and the plus-prescription, defined as
\begin{equation}
\left[f(x) \right]_+ = f(x) - \delta (1-x)\int_0^1 f(z) \, dz,
\end{equation}
has been used to regularize the divergent integral. 

An analogous function exists for the evolution of the gluon PDF
\begin{equation}
\frac{dg(x,Q^2)}{d\log Q^2} = \frac{\alpha_s}{2\pi}\int_x^1 \, \frac{dy}{y} \left[\sum_f q_f(y,Q^2)P_{gq}\left(\frac{x}{y} \right) + g(y,Q^2)P_{gg}\left(\frac{x}{y} \right) \right]
\end{equation}
where this time
\begin{equation}
\begin{cases}
P_{gq}(z) = C_F\left[ \frac{1 + (1-z)^2}{z}\right] \\
P_{gg}(z) = 2C_A\left[\frac{1 - z}{z} + \frac{z}{(1 - z)_+} + z(1-z) \right]
\end{cases}
\end{equation}
where $C_A$ is the Casimir operator for the adjoint representation of $SU(3)$.
 
\section{Fragmentation Functions}
Fragmentation functions play a role similar to that of PDFs, but rather than describing the possible initial states of the interaction, they describe the possible final states that may be observed.

To understand Fragmentation Functions, we refer to the \emph{semi-inclusive} DIS process, where we observe one hadron in the final state. The kinematic variables are the same as in (\ref{kinematics}), though we must now include the four-momentum of the final-state hadron
\begin{equation}
P_h^\mu = (E_h, \vec{P_h})
\end{equation} 
and one additional Lorentz invariant variable
\begin{equation}
z_h = \frac{P \cdot P_h}{P \cdot q} \overset{\mathrm{TRF}}{=} \frac{E_h}{\nu}
\end{equation}
which in the TRF represents the fraction of energy lost in the hadronization process of the observed particle. 

The cross section can again be calculated without difficulty:
\begin{equation}
\frac{d\sigma}{dx dy dz} = \frac{4\pi\alpha^2_S}{Q^4}\left(\frac{y}{2} + 1 - y\right)x \; \sum_f e_f^2 \phi_f(x) D_f(z)
\label{cross section FF}
\end{equation}
where the sum over all flavors includes the anti-quarks. In this way, the Fragmentation Function $D_f(z)$ represents the probability of finding a hadron with fraction $z$ of the available energy. The cross section remains factorized, allowing for information on the Fragmentation Functions by comparison with data, as neither they are calculable in QCD. Finally, it should be said that the Fragmentation Functions depend on the quark from which they originate, hence the index. 

\section{Jets}
Due to the non-perturbative nature of the hadronization process and the myriad of particles produced, it is not possible to calculate cross-sections for all of the different possible hadronic final states. Instead, \emph{jets} are considered in their place. Jets are, as the name suggests, groups of collimated particles which all originate from the same parent particle. They can be treated as singular objects, allowing for a notable simplification of the calculations. 

\subsection{Jet Definitions}
In order to be able to precisely calculate jet cross-sections and confront these with experimental data, it is necessarily to unambiguously \emph{define} what a jet is. There are multiple possible definitions of jets, and we will briefly some examples of these.  

\subsubsection{Cone Algorithms}
Historically, cone algorithms were the first class of jet algorithms introduced. The very first algorithm was used to classify jets in $e^+e^-$ collisions and depended on two arbitrary parameters, $\delta$ and $\epsilon$. If an event had a fraction of energy of at least $1 - \epsilon$ concentrated within two cones of half-angle $\delta$, then that even was said to contain two jets \cite{Sterman:1977wj}. The fact that the definition relied on the arbitrary parameters $\delta$ and $\epsilon$ meant that these had to be specified so that the predictions could be compared to data. This remains a general feature of jet algorithms to this day.

Cone algorithms have progressed in the years since their inception. Today, two of the most widely used code algorithms are iterative and fixed cone algorithms. In iterative cone algorithms, the direction of the jet is initially set by a particle $i$. All particles within distance $R_{ij}$
\begin{equation}
\Delta_{ij}^2 = (y_i - y_j)^2 + (\phi_i - \phi_j)^2 < R_{ij}^2 
\end{equation}
in the rapidity/azimuthal angle plane are then taken as part of the jet, and their momenta are summed. The result of this summation is then used as the new seed, and the process is iterated until the jet cone is stable. To fully specify the algorithm, one must choose how to take the seed, and what to do when jets overlap. 

Fixed cone algorithms function similarly. In this case, rather than iterating the cone direction, a cone is fixed around a seed, and that cone is called a jet. The particles within the radius $R_{ij}$ are assigned to the jet, and removed from the event record. The algorithm proceeds until all possible jets have been identified.

\subsubsection{$k_t$-Algorithms} 
$k_t$ algorithms are part of a family of algorithms known as sequential recombination jet algorithms. The $k_t$ algorithms use a momentum-weighted distance
\begin{equation}
\begin{cases}
d_{ij} = min(k_{ti}^{2p}, k_{tj}^{2p})\frac{\Delta_{ij}^2}{R^2} \\
d_{iB} = k_{ti}^{2p}
\end{cases}
\end{equation}
to establish which particles $j$ lie closest to the particle $i$. If $d_{ij}$ is less than the distance between $i$ and the beam $d_{iB}$, $i$ and $j$ are combined into a jet. This procedure iterates over all particles. 

These definitions also depend on two parameters: $p$ and $R$. There exist three noteworthy cases: $p = 1$ is known as the $k_t$ algorithm, and weighs the distance $d_{ij}$ using the square of the transverse momentum of the softer particle; $p = 0$ is known as the Cambridge-Aachen algorithm, and features no weighting; $p = -1$ is known as the anti-$k_t$ algorithm, and weights the distance using the inverse of the square of the transverse momentum of the harder particle. 

Due to the distance used, the anti-$k_t$ algorithm tends to cluster soft particles together with hard particles. This is a useful property as it tends to lead to jets centered about hard particles, and correctly recombines the soft radiation emitted from the hard seed together with that seed. If the hard particles are well-separated, this also leads to conical jets, as shown in Figure \ref{jet algos}.

\begin{figure}[hp]
\centering
\includegraphics[width=\textwidth]{ch2_images/jet_algos}
\caption{A representation of the jets formed from the same event using the three main $k_t$ algorithms as well as SISCone, a commonly used cone algorithm. The jets clustered using anti-$k_t$ are conical, and centered around the hard particles \cite{Cacciari:2008gp}.}
\label{jet algos}
\end{figure}

\subsection{Jet Cross Sections}

Once we have defined our jets, we can go on to calculate jet cross sections. Without going into too much detail, we will just illustrate a general feature which highlights the usefulness of jets. 

If we calculate the two-hadron semi-inclusive cross section for $e^+e^-$ into hadrons, we find that
\begin{equation}
\frac{d\sigma}{dydz_1dz_2} = N_c \frac{\pi \alpha^2}{Q^2} \left(1 + \cos\theta \right) \sum_f e_f^2 D_f(z_1) D_f(z_2),
\end{equation}
where again the sum over the flavors $f$ runs over both quarks and antiquarks. $N_c$ stands for the number of colors that can be produced, and $\theta$ is the angle between the momentum of the quarks produced and the momentum of the colliding leptons.

If rather than observing the two final-state hadrons we observe the jets surrounding them, the Fragmentation Functions are replaced by $\delta$-functions $\delta(1 - z_1)$ and $\delta(1 - z_2)$. After integrating, we find that
\begin{equation}
\frac{d\sigma^{jets}}{dy} = N_c \frac{\pi \alpha^2}{Q^2} \left(1 + \cos\theta \right) \sum_f e_f^2.
\end{equation}
\emph{This is exactly the QED cross section for $e^+e^-$ annihilation!} The use of jets allows us to notably simplify calculations and make accurate predictions, since we neither have to consider fully inclusive cross sections, nor a fully exclusive one.
 
\section{Higher Order Corrections}

A number of higher order corrections to the elementary cross section are possible. These include real and virtual corrections in all possible combinations. In QCD, we currently know how to treat corrections up to next-to-
next-to-next-to-leading order for a few select processes \cite{DUHR20162128, Anastasiou:2015vya}. In this section we will briefly discuss the importance of higher order corrections, as well as some challenges which arise during their calculation.

\subsection{Infrared and Collinear Divergences}
The matrix element for a radiative correction in QED and QCD is easily calculable using the Feynman rules. We will consider the example of the emission of a photon for $e^+e^- \rightarrow e^+e^-$ in QED, but all considerations also hold in QCD. The matrix element for all possible real emissions and all possible photon polarizations is found to be

\begin{figure}[ht]
\begin{subfigure}{.5\textwidth}
\centering
\includegraphics[scale=0.6]{ch2_images/real}
\label{real}
\caption{}
\end{subfigure}
\begin{subfigure}{.5\textwidth}
\centering
\includegraphics[scale=0.55]{ch2_images/virtual}
\label{virtual}
\caption{}
\end{subfigure}
\caption{Feynman diagrams representing real (a) and virtual (b) corrections to the final state of $e^+e^-$ annihilation in QED.}
\end{figure}

\begin{equation}
\frac{d\sigma}{d\Omega} = \left(\frac{d\sigma}{d\Omega} \right)_0 \frac{\alpha}{2\pi}\left[2\frac{p_- \cdot p_+}{(p_-\cdot k) (p_+ \cdot k)} - \frac{m_e^2}{(p_-\cdot k)^2} - \frac{m_e^2}{(p_+ \cdot k)^2} \right]\frac{d^3k}{\omega}
\label{infrared}
\end{equation}
where $\left(\frac{d\sigma}{d\Omega} \right)_0$ represents the tree-level cross section, without any emission, $p_-$ and $p_+$ represent the momenta of the outgoing electron and positron, and $k^\mu$ the momentum of the radiated photon. In the center-of-mass frame, where the kinematic variables can be expressed as
\begin{equation}
\begin{cases}
p_-^\mu = (E, 0, 0, \beta E)\\
p_+^\mu = (E, 0, 0, -\beta E)\\
k^\mu = (\omega, \omega \sin\theta\cos\phi, \omega\sin\theta\sin\phi, \omega\cos\theta)
\end{cases}
\end{equation}
where $\beta = \sqrt{1 - \frac{m_e^2}{E^2}}$ and $\omega$ is the energy of the photon, equation (\ref{infrared}) becomes
\begin{equation}
\frac{d\sigma}{d\Omega} = \left(\frac{d\sigma}{d\Omega} \right)_0 \frac{\alpha}{2\pi}\frac{1}{\omega}\frac{\beta^2\sin^2\theta}{(1-\beta^2\cos^2\theta)^2}d\omega \, d\cos\theta
\end{equation}
where we have already integrated in $d\phi$.

This expression is divergent in two cases: when the radiated particle is either very soft ($E \ll 1$) or collinear to the lepton from which it is emitted ($\theta \ll 1$). These cases are known as \emph{infrared} and \emph{collinear} divergences, respectively, and they are problematic as they prohibit the calculation of a finite cross section at next-to-leading order. 

Integrating in $d\cos\theta$ resolves the collinear divergence
\begin{equation}
\frac{d\sigma}{d\Omega} = \left(\frac{d\sigma}{d\Omega}\right)_0 \frac{1}{\omega} \frac{2\alpha}{\pi}\left[\frac{\beta^2 +1}{2\beta} \ln \left( \frac{1 + \beta}{1 - \beta}\right) +1 \right]d\omega
\end{equation}
which, when we consider that $s \gg m^2$ simplifies to
\begin{equation}
\frac{d\sigma}{d\Omega} = \left(\frac{d\sigma}{d\Omega}\right)_0 \frac{1}{\omega} \frac{2\alpha}{\pi}\left[\ln\frac{s}{m_e^2} - 1 \right] d\omega.
\label{cross section divergence}
\end{equation}
Lastly, we must integrate over all energies $d\omega$. However, because of finite detector resolution, the experimental cross section can only be sensitive to those photons radiated over a threshold $\Delta E$, the experimental cross section contains two parts
\begin{equation}
\left( \frac{d\sigma}{d\Omega}\right)_{exp} = \left( \frac{d\sigma}{d\Omega}\right)_{elastic} + \left( \frac{d\sigma}{d\Omega}\right)_{\omega < \Delta E}
\label{cross section decomposition}
\end{equation}
one corresponding to the tree-level cross section, and the other with a radiative correction with $\omega < \Delta E$. When we integrate over $d\omega$, the second term on the right-hand size diverges as can be seen from (\ref{cross section divergence}).

Thankfully, a brilliant solution to this problem exists. If we assign a mass $\lambda$ to the radiated photon, the (\ref{cross section divergence}) becomes
\begin{equation}
\left( \frac{d\sigma}{d\Omega}\right)_{exp} = \left( \frac{d\sigma}{d\Omega}\right)_{elastic} + \left( \frac{d\sigma}{d\Omega}\right)_{\omega < \Delta E}\ln \frac{\Delta E}{\lambda}.
\end{equation}
If we also consider the virtual corrections and calculate the relative cross section, which also contributes to (\ref{cross section decomposition}), we find

\begin{equation}
\frac{d\sigma}{d\Omega} = \left(\frac{d\sigma}{d\Omega}\right)_0 \frac{1}{\omega} \frac{2\alpha}{\pi}\left[\ln\frac{s}{m_e^2} - 1 \right] \ln \frac{\lambda}{E} d\omega
\end{equation}
We can sum the real and virtual corrections, and find a total cross section independent from $\lambda$! The Kinoshita–Lee–Nauenberg theorem guarantees that this cancellation occurs at all orders in perturbation theory \cite{Kinoshita:1962ur}. We can now safely take the limit of $\lambda \rightarrow 0$ and integrate over $\omega$.

\subsection{Infrared and Collinear Safety for Jets}
As mentioned in the previous section, infrared and collinear (IRC) divergences will, in theory, exactly cancel. In practice, however, this is not always the case. Depending on the type of jet algorithm used, it may happen that the cancellation breaks and an infinite cross section is calculated. Obviously, this is a problem since the measured cross section is by definition finite. It is therefore important from a theoretical standpoint to define in a way that is IRC safe. 

An observable $\mathcal{O}$ is said to be IRC safe, respectively, when the following properties are satisfied:
\begin{gather}
\mathcal{O}(X; p_1, \dots, p_n, p_{n+1} \rightarrow 0) \rightarrow  \mathcal{O}(X; p_1, \dots, p_n ) \label{IRC safety}\\
\mathcal{O}(X; p_1, \dots, p_n \parallel p_{n+1}) \rightarrow \mathcal{O}(X; p_1, \dots, p_n),
\end{gather}
i.e.\ when the observable reduces to that of $n$ particle case, in absence of the soft or collinear emission. An example of infrared safe and unsafe jets is shown in Figure \ref{Infrared safe jet}, while Figure \ref{collinear safe jet} illustrates an example of collinear safe and unsafe jets.

The anti-$k_t$ algorithm is an example of a jet algorithm which is IRC safe. Due to its weighting, when a soft particle is emitted, this will tend to cluster together with the hard center of the jet. On the other hand, when a particle is emitted collinear to another, the distance $\Delta_{ij}$ will be very small, and it will again be clustered together with its parent particle.


\begin{figure}[p]
\centering
\includegraphics[width=\textwidth]{ch2_images/ir_safe}
\caption{A schematic representation of an IR unsafe jet. The emission of a soft gluon changes the event from a two-jet event to a one-jet event due to its effect on the clustering algorithm \cite{Salam:2010nqg}.}
\label{Infrared safe jet}
\end{figure}

\begin{figure}[p]
\centering
\includegraphics[width=\textwidth]{ch2_images/collinear_safe}
\caption{A schematic representation of a collinear safe and unsafe jet. The blue lines represent particles, and the length corresponds to the particle's transverse momentum. The horizontal axis represents rapidity. If the emission of a collinear gluon leads to the formation of a second jet, the cancellation is broken and the cross section diverges, as shown on the right. If, on the other hand, it and its parents are treated as a single particle by, for example, summing their momenta, the emission satisfies the definition given by Equation (\ref{IRC safety}) \cite{Salam:2010nqg}.}
\label{collinear safe jet}
\end{figure}

\section{Conclusions}
Over the past few decades, colliders have allowed us to study, understand, and ultimately build the Standard Model. Notable achievements include  the discovery of the $Z$ boson through the observation of weak neutral currents at Gargamelle in 1973 \cite{GargamelleNeutrino:1973jyy, HASERT1973121}, the discovery of the $W$ boson by the UA1 and UA2 collaborations in 1983 \cite{UA1:1983crd, UA2:1983tsx}, and the confirmation of the existence of three generations of matter at LEP in 1998 \cite{L3:1998uub}. At the end of the '90s, only one key ingredient of the model had yet to be found: the Higgs boson.

\nocite{Mangano:2018sfp}
\end{document}


%Jet production and kinematic variables
%Next chapter:
%VH production
%Higgs decays
%Conclusion->famous plot describing all of physics at colliders
