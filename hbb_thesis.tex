\documentclass[10pt,a4paper]{book}
\usepackage[utf8]{inputenc}
\usepackage[english]{babel}
\usepackage{amsmath}
\usepackage{amsthm}
\usepackage{mathtools}
\usepackage{array}
\usepackage{booktabs}
\usepackage{gensymb}
\usepackage{slashed}
\usepackage{physics}
\usepackage{bbold}
\usepackage{stackengine}
\usepackage{amsfonts}
\usepackage{amssymb}
\usepackage{graphicx}
\usepackage{geometry}
\usepackage{pdfpages}
\usepackage{hyperref}
\usepackage{imakeidx}
\usepackage[toc]{appendix}
\usepackage{url}
\usepackage[numbers,sort&compress]{natbib}
\usepackage{subcaption}
\newtheorem*{theorem*}{Theorem}
\newcolumntype{L}{>{$}c<{$}}
\newcommand\todo[1]{\textcolor{red}{#1}}
\newenvironment{abstract}{}{}
\def\code#1{\texttt{#1}}


\usepackage{blindtext}

\usepackage{subfiles} % Best loaded last in the preamble

\title{Tesi}
\author{Alberto Lorenzo Rescia}
\date{\today}

\makeindex

\begin{document}

\includepdf[page=-]{frontespizio.pdf}

\chapter*{Abstract}
$\indent$ In recent years, $H \rightarrow b\overline{b}$ decays in association with $VH$ production have been observed at the LHC. For a Higgs mass of 125 GeV, this decay channel has by far the largest branching ratio. It is therefore imperative to be able to correctly identify the resulting b-jets from this decay as having originated from the Higgs, as opposed to one of the many processes constituting the QCD background.

To this aim, we focus on the problem of distinguishing b-jets originating from the decay of a color singlet from those originating from a color octet by means of a combination of color-sensitive variables introduced in literature. 
We simulate 13 TeV pp collisions at the LHC and evaluate color-sensitive observables before and after a fast simulation of the detector response. These variables are used to feed several machine learning algorithms. Results are given in terms of performance of single variables and of their combination.\\

\bigskip
Negli anni recenti, il decadimento $H\rightarrow b\overline{b}$ in associazione con la produzione $VH$ \`{e} stato osservato al LHC. Per una massa dell'Higgs pari a 125 GeV, questo canale di decadimento ha di gran lunga il pi\`{u} grande branching ratio: \`{e} quindi importante essere in grado di identificare correttamente i b-jet risultanti da questo decadimento come originatisi dal bosone di Higgs, differentemente da uno dei tanti processi che costituiscono il fondo di QCD.

A questo scopo, ci focalizziamo sul problema di distinguere b-jet originatisi da un decadimento di un singoletto di colore da quelli originatisi da un ottetto di colore attraverso una combinazione di variabili sensibili al colore introdotte in letteratura. Simuliamo collisioni protone-protone a 13 TeV al LHC e valutiamo le osservabili sensibili al colore prima e dopo una simulazione ``fast'' della risposta del rivelatore. Queste variabili sono poi utilizzate per allenare diversi algoritmi di machine learning. I risultati sono dati in termini della performance delle singole variabili e della loro combinazione. 

\tableofcontents

\chapter*{Introduction}
\addcontentsline{toc}{chapter}{Introduction} 

The discovery of the Higgs boson in 2012 represented a turning point in our understanding of the Standard Model of particle physics. This discovery was so remarkable that it earned Peter Higgs and Fran\c{c}ois Englert the 2013 Nobel Prize in Physics. Finally, after years of searching, the missing degree of freedom of the Standard Model was found, and physicists could begin to undertake the task of rigorously testing all properties of this particle.

The Higgs boson could give us clues regarding the possibility of new physics. For example, several models of physics beyond the Standard Model include extensions containing expanded Higgs sectors. In addition to this, the currently know Higgs sector could provide insight to the nature of dark matter, since, if dark matter is a particle, it is natural to assume that it could couple to the Higgs boson.  Finally, it cannot be excluded that a new discovery could stem from of an unexpected anomaly or deviation from the Standard Model expectations. It is therefore imperative to be able to identify as many events containing the Higgs as possible, in order to have sufficient statistics to find and claim any eventual new discovery.

The task of identifying events containing the Higgs is rendered difficult by the overwhelming QCD background. To further complicate this matter, the most likely decay channel of the Higgs, $H\rightarrow b\overline{b}$, is characterized by a final state which is far more likely to be produced by the decay of a gluon. In this thesis, we illustrate a framework that we have developed to try to overcome this barrier by recognizing and distinguishing the two different color configurations of the Higgs boson and the gluon.

In Chapter 1, we begin by contextualizing the role of the Higgs boson within the Standard Model. In particular, we describe from a theoretical perspective the principal interactions of the Standard Model, and show how they come together to build a complete theory.

Chapter 2 focuses on the physics of colliders. Over the past 50 years, colliders have been one of our main tools in understanding and building the Standard Model, and they continue to play a fundamental part in our understanding. We describe, again from a theoretical viewpoint, the physics behind particle collisions, and focus in particular on the role of jets, which represent a crucial bridge between theory and experiment.

Chapter 3 discusses Higgs physics from an experimental point of view. We describe the main production and decay mechanisms, examine the infamous 2012 discovery, and briefly describe the role that the Higgs sector will play moving forward.

Finally, in Chapter 4, we discuss the framework that we have developed to distinguish decays of color singlets from those of color octects. We begin by describing eight high-level, color-sensitive variables which can be measured at colliders. We proceed by describing our simulation of proton-proton collisions at 13 TeV, and the analysis used to extract these variables. We conclude by discussing the details of the architectures of the machine learning algorithms used to identify the two different color configurations, and illustrate our findings.

\chapter{The Standard Model}
\subfile{ch1}

\chapter{Collisions at the LHC}
\subfile{ch2}

\chapter{Higgs Physics}
\subfile{ch3}
%Cite measurement of Hbb at ATLAS and CMS
%Discovery of Higgs at CMS and ATLAS
%Update Feynman diagrams
%Aggiungere Properties of the Higgs as measured at colliders
%mass, width, ecc. 

\chapter{A Study of Color Sensitive Observables}
\subfile{ch4}

\chapter*{Conclusions and Outlook}
\addcontentsline{toc}{chapter}{Conclusions and Outlook} 

In the framework, we have developed a machine learning based method capable of distinguishing decays originating from color singlets from those originating from color octects. Specifically, we have trained a Multilayer Perceptron and a Boosted Decision Tree on eight high-level, color-sensitive variables in order to develop a color tagger.

Given the high value of the area under the ROC found when using simulated particle-level data and simulated data including detector effects (about 0.82 in the first case and 0.77 in the second), we can conclude that the method is not only effective in theory, but also shows promising prospects for application to experiment. A paper summarizing the main results is under preparation.

The ATLAS Xbb Tagger group has already expressed interest in this work. In the future, we plan to run a full simulation of the ATLAS detector, to then test the performance of the method using real data. At that point, the method can be integrated into the existing b-tagging algorithms used at ATLAS, and can provide additional information regarding the partonic origin of jets.

We also hope to improve this method by introducing other variables. In particular, we would like to include the Lund Jet Plane \cite{Dreyer:2018nbf}, a phase-space representation of jet substructure which has already been successfully measured at ATLAS \cite{ATLAS:2020bbn}. Preliminary results are extremely encouraging. 

In the coming years, the increased statistics as a result of the LHC luminosity upgrade promise to provide us with more insight on the properties of the Higgs. With luck, precision studies led by both theoretical and experimental efforts will lead to new discoveries or to some deviation from the current understand of particle physics, which may be the footprint of a new, previously unforeseen sector of physics 

%Appendix where we discuss NN and BDT
%Chapter about ATLAS

\begin{appendices}
\chapter{Machine Learning}
\label{Appendix}
\subfile{mlp}
\end{appendices}

\bibliographystyle{unsrt}
\bibliography{hbb_thesis.bib}

\end{document}