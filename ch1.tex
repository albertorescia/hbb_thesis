\documentclass[10pt,a4paper]{book}
\usepackage[utf8]{inputenc}
\usepackage[english]{babel}
\usepackage{amsmath}
\usepackage{mathtools}
\usepackage{array}
\usepackage{booktabs}
\usepackage{gensymb}
\usepackage{slashed}
\usepackage{physics}
\usepackage{bbold}
\usepackage{stackengine}
\usepackage{amsfonts}
\usepackage{amssymb}
\usepackage{graphicx}
\usepackage{geometry}
\usepackage{pdfpages}
\usepackage[numbers,sort&compress]{natbib}


\newtheorem{theorem}{Theorem}[section]
\newcolumntype{L}{>{$}c<{$}}
\newcommand\todo[1]{\textcolor{red}{#1}}


\begin{document}

Since 2009, the Large Hadron Collider (LHC) has been exploring particle physics at the high energy frontier in the hope of discovering new physics, including, but not limited to, the Higgs boson, supersymmetry, flavor violations, and much more. At the basis of all physics at the LHC lies the Standard Model. This theory describes all fundamental interactions involving the electroweak and strong forces, as well as the fields which partake in these interactions. In this chapter, we will briefly illustrate the main features of this theory and show how together they paint a complete picture of our current understanding of elementary particle physics.


%Gauge Invariance
%Lagrangians of the SM fields
%EW unification
%EW symmetry breaking
%The (full) Lagrangian of the SM

\section{An Overview of the Theory}
%Delineo la struttura del modello standard
%Scrivo la lagrangiana dell'interazione em e forte
The Standard Model of particle physics is composed of two sectors: the matter fields and gauge fields. The matter fields are fermionic fields whose excitations lead to the particles which make up ordinary matter, i.e. quarks and leptons, and their corresponding antiparticles. The gauge fields, on the other hand, are bosonic fields which arise from symmetries of the model and describe the force-carrying particles, specifically the photon ($\gamma$), gluons (g), W$^{\pm}$ and Z, as well as the Higgs boson, H.

The matter fields are organized in three generations, as shown in Figure \ref{Standard Model particles}.
\begin{figure}
\centering
\includegraphics[scale=0.3]{ch1_images/standard_model}
\caption{The ``Periodic Table'' of the Standard Model, showing the division between matter fields and gauge bosons, as well as the three different generations of the matter fields. Properties such as mass, charge and spin are also shown \cite{sm_particles}.}
\label{Standard Model particles}
\end{figure}
The particles in the three generations all have the same properties, with the sole exception of mass which increases with each generation. Neutrinos might be the exception to this rule, since their masses are not yet known to a useful degree of precision. 

The properties of the matter fields determine the interactions in which they participate. All fermions can interact via the weak force. The electrically charged particles participate in electromagnetic interactions, and quarks, the only particles carrying color charge, can also interact via the strong nuclear force.

The interactions with these forces are mediated by the exchange of the gauge bosons. Photons are responsible for electromagnetic interactions, gluons for strong interactions, and the W$^\pm$ and Z for weak interactions. 

The strength of the interactions are determined by the value of the coupling constants, $\alpha$ for electromagnetic interactions, $\alpha_W$ for weak interactions and $\alpha_S$ for strong interactions. Despite their name, the coupling constants are known to run, i.e. they depend on the energy scale $Q^2$ of the process. Figure \ref{running coupling constants} shows how the coupling constants vary with energy. As can be seen, $\alpha$ and $\alpha_W$ increase as $Q^2$ increases, while $\alpha_S$ decreases with increasing energy, a property known as asymptotic freedom. 

\begin{figure}
\centering
\includegraphics[scale=0.5]{ch1_images/running}
\caption{The $\beta$-functions for the three coupling constants included in the Standard Model \cite{wiki:xxx}.}
\label{running coupling constants}
\end{figure}

The running of the coupling constants is determined by the $\beta$-function, a renormalization group equation which can be solved in perturbation theory. At one-loop, the electromagnetic $\beta-$function is given by \cite{pinkbook}
\begin{equation}
\beta(\alpha) = \frac{\alpha^2}{3\pi} + \mathcal{O}(\alpha^3),
\end{equation}
while the $\beta$-function for the strong interaction can be written as
\begin{equation}
\beta(\alpha_S) = -\left(11 - \frac{2n_f}{3} \right)\frac{\alpha_S^2}{2\pi} + \mathcal{O}(\alpha_S^3)
\end{equation}

This inverse dependence on energy is a direct consequence of the non-Abelian nature of the strong nuclear force, as well as the number of flavors of quarks, $n_f$ which enters the loops.  

As mentioned, the gauge bosons originate from symmetries of the Standard Model. Specifically, the Standard Model Lagrangian is invariant under a local $SU(3)_C \otimes SU(2)_L \otimes U(1)_Y$ gauge symmetry. This means that, when a field $\psi$ transforms under the action of this group, i.e. when
\begin{equation}
\psi \rightarrow \psi^\prime = U(x)\psi,
\end{equation}
the Lagrangian remains unchanged. 

This is an example of invariance under a continuous symmetry. Per N\"{o}ther's theorem, these symmetries lead to conserved quantities, namely the conservation of color, weak isospin and hypercharge in every point of spacetime. There are a number of conserved quantities which arise from symmetries: momentum from translational symmetries, angular momentum from rotational symmetries, and electric charge from a $U(1)_{EM}$ symmetry, just to name a few.

There are also a number of discrete symmetries present in the model. The most significant of these are \emph{parity}, which inverts the spatial coordinates of a quantity
\begin{equation}
\hat{P}\psi(\vec{x}, t) = \psi(-\vec{x}, t),
\end{equation}
\emph{charge symmetry}, which exchanges all particles with the corresponding antiparticles,
\begin{equation}
\hat{C}\psi(x) = i\gamma^2\psi^*(x)
\end{equation}
and \emph{time-reversal symmetry} inverts the sign of the temporal coordinate of a quantity
\begin{equation}
\hat{T}\psi(\vec{x}, t) = \psi(\vec{x}, -t).
\end{equation}
These symmetries are not individually present in all processes. For example, QED and QCD are both invariant under transformations of parity. These theories involve vector currents of the form $j^\mu = \overline{\psi}\gamma^\mu \psi$. These transform as
\begin{equation}
\begin{cases}
\hat{P}j^{0} = \overline{\psi}\gamma^{0}\gamma^{0}\gamma^{0}\psi = \overline{\psi}\gamma^{0}\psi\\
\hat{P}j^{i} = \overline{\psi}\gamma^{0}\gamma^{i}\gamma^{0}\psi = -\overline{\psi}\gamma^{i}\gamma^{0}\gamma^{0}\psi = -\overline{\psi}\gamma^{i}\psi = -j^{i}.
\end{cases}
\end{equation}
When we consider the matrix element $\mathcal{M}$ for one of these interactions, which is proportional to the scalar product of two currents, we find that
\begin{equation}
\hat{P}j_{1} \cdot \hat{P}j_{2} = j^{0}_{1}j^{0}_{2} - \sum_{i}\left(-j^{i}_{1}\right)\left(-j^{i}_{2}\right) = j_{1} \cdot j_{2},
\end{equation}
i.e. the interaction is invariant under parity transformations.
On the other hand, weak interactions famously show parity violations \cite{PhysRev.105.1413}. This is due to the fact that the currents involved are of type V-A, i.e. of the form $j^\mu \propto \overline{\psi}\gamma^\mu\left(1 - \gamma_5\right)\psi$. This form mixes a vector current, with an axial vector current of the form $j^\mu_A = \overline{\psi}\gamma^\mu\gamma^5\psi$. In this case, when we consider the matrix element
\begin{equation}
\mathcal{M} \propto j_{1} \cdot j_{2} = g^{2}_{V}(j_{V} \cdot j_{V}) + g^{2}_{A}(j_{A} \cdot j_{A}) + g_{V}g_{A}(j_{V} \cdot j_{A} + j_{A} \cdot j_{V})
\end{equation}
we will find three terms which conserve parity, and one parity-violating term, since
\begin{equation}
\hat{P}j_{V} \cdot \hat{P}j_{A} = -j_{V} \cdot j_{A}.
\end{equation}
Overall, only the combination of charge, parity, and time-reversal symmetry is expected to be conserved in all Standard Model interactions.
%Questa parte ci deve andare, però è noioso scriverla-
%Introduzione teoria di Dirac, e altre lagrangiane
%Fenomenologia modello standard
%Teorema di Noether e simmetrie modello standard

\section{Gauge Symmetries}
%Questa parte invece è molto più divertente.
\subsection{QED Lagrangian}
We shall now begin to construct the Standard Model Lagrangian. Let us start by considering the free Lagrangian for a massive fermion field, given by the Dirac Lagrangian

\begin{equation}
\mathcal{L}_D = \overline{\psi}(i\slashed{\partial} - m )\psi,
\label{Dirac lagrangian}
\end{equation}

where $\psi$ is the fermion field, $\overline{\psi}$ its Dirac adjoint, and, given the Dirac matrices $\gamma^\mu$, $\slashed{\partial}$ is the del operator in Feynman slash notation.  It is easy to show that this Lagrangian is invariant under transformations of the type

\begin{equation}
\psi \rightarrow \psi^\prime = \exp(ie\alpha)\psi
\label{global gauge symmetry}
\end{equation}
where $e$ is a parameter which represents the coupling constant and $\alpha$ is, for now, a parameter independent of the space-time coordinate $x$.
In fact, the analogous transformation for the adjoint field $\overline{\psi}$ is
\begin{equation}
\overline{\psi} \rightarrow \overline{\psi^\prime} = [\exp(ie\alpha)\psi]^\dagger \gamma^0 = \psi^\dagger \exp(-ie\alpha) \gamma^0 = \overline{\psi}\exp(-ie\alpha)
\end{equation}
since the operator $\exp(-ie\alpha)$ commutes with $\gamma^0$. When applied to the whole Lagrangian, the transformation has the overall effect of leaving it unchanged:

\begin{equation}
\mathcal{L}_D \rightarrow \mathcal{L}^\prime_D= \overline{\psi^\prime}(i\slashed{\partial} - m)\psi^\prime = \overline{\psi}(i\slashed{\partial} - m )\psi = \mathcal{L}_D.
\end{equation}
Since the Lagrangian is unchanged, so too are the equations of motion. The transformed fields will therefore have the same dynamics. 

We have just shown that the Dirac Lagrangian is invariant under a \emph{global} $U(1)$ gauge symmetry in charge space. 

At this point, if we want to construct the QED Lagrangian, we must add the kinetic term describing the free photon field
\begin{equation}
\mathcal{L}_{kin} = -\frac{1}{4}F^{\mu\nu}F_{\mu\nu}
\label{kinetic term}
\end{equation}
where $F_{\mu\nu} = \partial_\mu A_\nu - \partial_\nu A_\mu$, as well as a term describing the interaction between the two fields. We can do this in two ways. 

\subsubsection{Minimal Coupling}

The first, more direct, prescription calls for applying the minimal coupling rule. This requires substituting the four-momentum of the fermion, which we will take to be an electron, with an expression which includes the electromagnetic potential

\begin{equation}
p_\mu \rightarrow p_\mu - eA_\mu
\end{equation}
and the coupling constant $e$. Quantum mechanically, this corresponds to substituting the del operator in the Lagrangian. The Lagrangian thus becomes

\begin{equation}
\label{QED lagrangian}
\mathcal{L}_{QED} = -\frac{1}{4}F^{\mu\nu}F_{\mu\nu} + \overline{\psi}(i\slashed{\partial} - e\slashed{A} - m)\psi = \mathcal{L}_{kin} + \mathcal{L}_D - e\overline{\psi}\gamma^\mu \psi A_\mu.
\end{equation}
This Lagrangian is still invariant under the same global gauge symmetry as before. 

We would like, however, to impose a more stringent symmetry requirement: a \emph{local} gauge symmetry dependent on the space-time coordinate $x$. Whereas a global symmetry establishes the conservation of a conserved quantity, e.g. electric charge, in any closed system, the local symmetry imposes the same requirement in each point $x$. 

If we promote the gauge symmetry to a local symmetry, i.e. we apply the transformation 
\begin{equation}
\psi \rightarrow \psi^\prime = \exp[ie\alpha(x)]\psi,
\label{fermi-field transformation}
\end{equation} 
we find that the Lagrangian is no longer invariant under this transformation due to the action of the derivative. In fact, ignoring the terms which remain invariant, we find that
\begin{equation}
\mathcal{L}^\prime = i\overline{\psi^\prime} \slashed{\partial} \psi^\prime = i\overline{\psi^\prime} \exp[-ie\alpha(x)] \gamma^\mu \partial_\mu \lbrace \exp[ie\alpha(x)]\psi \rbrace = i\overline{\psi}\slashed{\partial}\psi - e\overline{\psi}\gamma^\mu\psi\partial_\mu\alpha(x).
\label{local gauge}
\end{equation}

We can use a trick to reobtain the gauge invariance. We know that the electromagnetic tensor $F^{\mu\nu}$ is gauge invariant. This means that if $A_\mu$ undergoes the transformation

\begin{equation}
A_\mu \rightarrow A_{\mu}^\prime = A_\mu + \partial_\mu f(x)
\label{gauge field transformation}
\end{equation}
where $f(x)$ is a function such that $\square f = 0$, then

\begin{equation}
F_{\mu\nu} \rightarrow F_{\mu\nu}^\prime = \partial_\mu (A_\nu + \partial_\nu f) - \partial_\nu (A_\mu + \partial_\mu f) = F_{\mu\nu}.
\end{equation}
If we choose $f$ opportunely, we can cancel out the extra term which appears in (\ref{local gauge}) with the last term in (\ref{QED lagrangian}). Specifically, the choice $f(x) = -\alpha(x)$ satisfies our request.
Therefore, by combining the transformations (\ref{fermi-field transformation}) and (\ref{gauge field transformation}), we can obtain an invariant Lagrangian.

\subsubsection{Gauge Principle}

The second, more general, way of adding an interaction term to the Lagrangian is by the Gauge Principle. This principle describes a protocol through which we can obtain the dynamics of QED, or any field theory, starting from the global gauge transformation (\ref{global gauge symmetry}).

We start, once again, from the Lagrangian (\ref{Dirac lagrangian}). Having identified the global gauge symmetry of the Lagrangian and having promoted it to a local symmetry, we define the covariant derivative as 
\begin{equation}
D_\mu \; \dot{=} \; \partial_\mu + ieA_\mu.
\end{equation}
We then require that the term $D_\mu \psi$ transforms as the field $\psi$ itself
\begin{equation}
D_\mu \psi \rightarrow (D_\mu \psi)^\prime = D_\mu^\prime \psi^\prime = \lbrace \partial_\mu + ieA^\prime_\mu \rbrace \psi^\prime = \exp[ie\alpha(x)]D_\mu\psi.
\end{equation}
By developing the equality, we find that $A^\prime_\mu$ must be given by

\begin{equation}
A^\prime_\mu = A_\mu - \partial_\mu \alpha(x)
\end{equation}
in order for the Lagrangian to remain invariant.

We can then use the covariant derivative to build a term which describes the free propagation of the field $A_\mu$. We do this by computing the commutator. With some basic algebra, we find that 
\begin{equation}
[D_\mu, D_\nu] = ie \lbrace \partial_\mu A_\nu - \partial_\nu A_\mu \rbrace \equiv ie F_{\mu\nu},
\end{equation}
where $F_{\mu \nu}$ is now a generic tensor of the field $A_\mu$. We thus have
\begin{equation}
F_{\mu \nu} = -\frac{i}{e}[D_\mu, D_\nu].
\end{equation}
We can then use the field tensor to construct a normalized, gauge invariant Lorentz scalar which will necessarily take the form (\ref{kinetic term}).

We have thus arrived at the QED Lagrangian in a general way, without assuming any prior knowledge about the field $A_\mu$. 

\subsection{QCD Lagrangian}

Armed with the gauge principle, it is now straightforward to derive the QCD Lagrangian. We must note, however, that a few complications arise from the fact that we are now dealing with a non-abelian gauge theory, i.e.\ a theory whose symmetry group is non-commutative. For a general Yang-Mills theory, the gauge group is $SU(N)$, but in QCD we will be working with $N = 3$. 

The Dirac field for the quark can be indicated as $q_f^\alpha$ where $f$ is the flavor index and $\alpha$ is the color index. We know that each flavor comes in three colors, so we can group the fields for each flavor in a three-component vector
\begin{equation}
q_f = 	\begin{bmatrix} 	
		q^1_f \\ 
		q^2_f \\
		q^3_f \\
		\end{bmatrix}.
\end{equation}
We can thus write the free Lagrangian for the quarks as 
\begin{equation}
\mathcal{L}_D = \sum_f\overline{q}_f\left(i\slashed{\partial} - m_f\right)q_f
\end{equation}
where $m_f$ is a parameter representing the quark mass and $(i\slashed{\partial} - m_f)$ is a 3-dimensional diagonal matrix. The quark mass $m_f$ must be understood as a free parameter of the Lagrangian since it is not directly measurable, as free quarks do not exist in nature.

The Lagrangian is invariant under the following global gauge transformations in color space:
\begin{equation}
q_f \rightarrow (q_f)^\prime = \exp[i\theta_a\frac{\lambda^a}{2}]q_f
\end{equation}
where $\theta_a$ is a parameter and a = $1, \dots, 8$ since, in general, the fundamental representation of $SU(N)$ has $N^2 - 1$ generators. $\lambda^a$ represents the Gell-Mann matrices, which in the fundamental representation of $SU(3)$ can be written as
\begin{equation*}
\lambda^1 = \begin{bmatrix} 0 & 1 & 0 \\ 1 & 0 & 0 \\ 0 & 0 & 0 \end{bmatrix}, \qquad
\lambda^2 = \begin{bmatrix} 0 & -i & 0 \\ i & 0 & 0 \\ 0 & 0 & 0 \end{bmatrix}, \qquad
\lambda^3 = \begin{bmatrix} 1 & 0 & 0 \\ 0 & -1 & 0 \\ 0 & 0 & 0 \end{bmatrix},
\end{equation*}
\begin{equation}
\lambda^4 = \begin{bmatrix} 0 & 0 & 1 \\ 0 & 0 & 0 \\ 1 & 0 & 0 \end{bmatrix}, \qquad
\lambda^5 = \begin{bmatrix} 0 & 0 & -i \\ 0 & 0 & 0 \\ i & 0 & 0 \end{bmatrix},
\end{equation}
\begin{equation*}
\lambda^6 = \begin{bmatrix} 0 & 0 & 0 \\ 0 & 0 & 1 \\ 0 & 1 & 0 \end{bmatrix},\qquad
\lambda^7 = \begin{bmatrix} 0 & 0 & 0 \\ 0 & 0 & -i \\ 0 & i & 0 \end{bmatrix}, \qquad
\lambda^8 = \frac{1}{\sqrt{3}} \begin{bmatrix} 1 & 0 & 0 \\ 0 & 1 & 0 \\ 0 & 0 & 2
\end{bmatrix}.\\
\end{equation*}
The Gell-Mann matrices also allow us to define the structure constant of $SU(3),$ $f_{abc}$
\begin{equation}
\left[\frac{\lambda_a}{2}, \, \frac{\lambda_b}{2}\right] = if_{abc}\frac{\lambda_c}{2}. 
\end{equation}

We can now proceed with the gauge principle. We define the covariant derivative as
\begin{equation}
D_\mu = \partial_\mu +ig_s\frac{\lambda_a}{2}G^a_\mu
\end{equation}
where we have introduced the strong coupling constant $g_s$ and 8 spin-1 vector fields $G^a_\mu$. These are the gluon fields.
We now promote $\theta_a$ to $\theta_a(x)$ and require that $D_\mu q_f$ transform as $q_f$ so as to fix the interaction term between the quarks and the gauge bosons. We find that
\begin{equation}
G^a_\mu \rightarrow (G^a_\mu)^\prime = G^a_\mu - \frac{1}{g_s}\partial_\mu\theta^a(x) - f^{abc}\partial_\mu\theta_b(x)G_{\mu c}.
\end{equation}
Last but not least, using the relation
\begin{equation}
-\frac{i}{g_s}[D_\mu, D_\nu] =  \frac{\lambda_a}{2}G^a_{\mu \nu}
\end{equation}
we can define the gluon tensor field
\begin{equation}
G^a_{\mu\nu} = \partial_\mu G^a_\nu - \partial_\nu G^a_\mu - g_s f^{abc}G_{\mu b}G_{\nu c}
\end{equation}
which we use to construct the gauge-invariant kinetic term with proper normalization.
We thus find that the complete Lagrangian takes the following form:
\begin{equation}
\mathcal{L}_{QCD} = -\frac{1}{4}G^a_{\mu\nu}G^{\mu\nu}_a + \sum_f\overline{q}_f\left(i\slashed{D} - m_f\right)q_f.
\end{equation}

\section{Electroweak Unification}
\subsection{Lagrangian for Pure Weak Interactions}
The gauge principle can also be applied to weak interactions. In this case, the fundamental symmetry is $SU_L(2)$, labelled as such because it only applies to left-handed particle states or right-handed anti-particle states. From here on we will consider only particle states, though analogous considerations hold for anti-particle states. 

The symmetry acts on a weak-isospin doublet, e.g.\
\begin{equation}
\label{weak doublet}
\psi_L = \begin{bmatrix}
\nu_\ell \\
\ell^-
\end{bmatrix}_L
\end{equation}
composed of a left-handed neutrino and a lepton, or in general the left-handed states of any two fermions belonging to the same generation\footnote{Quark mixing slightly complicates this.}. Corresponding leptonic right-handed states $\ell^-_R$ are placed in a singlet state, and right-handed neutrino states are not considered as they have not been observed in nature \cite{PhysRev.109.1015}.

As before, by applying the local gauge transformations
\begin{gather}
\label{weak gauge transformations}
\psi_L \rightarrow \psi_L^\prime = \exp\left[i \frac{\tau_j}{2}\alpha^j (x) \right]\psi_L \\
\ell_R \rightarrow \ell_R^\prime = \ell_R
\end{gather}
where i=1,2,3 and $\tau^i$ are the generators of $SU(2)$, usually chosen to be the Pauli spin matrices

\begin{equation}
\tau^1 = \begin{bmatrix}
0 & 1  \\
1 & 0
\end{bmatrix}, \; \;
\tau^2_2 = \begin{bmatrix}
0 & -i  \\
i & 0
\end{bmatrix}, \; \;
\tau^3 = \begin{bmatrix}
1 & 0 \\
0 & -1
\end{bmatrix},
\end{equation}
we can derive the full Lagrangian describing weak interactions
\begin{equation}
\mathcal{L}_W = -\frac{1}{4}W^i_{\mu\nu}W^{\mu\nu}_i + \overline{\psi}_L(i\slashed{D} - m_i)\psi_{L} +  \ell_{R}(i\slashed{\partial} -m_i)\ell_{R}
\end{equation}
where $D_\mu = \partial_\mu +ig \frac{\tau_j}{2}W^j_\mu(x)$ and $g$ is the weak coupling constant.

The fields $W^1_\mu(x), W^2_\mu(x),$ and $W^3_\mu(x)$ in their raw form are not sufficient to describe the observed phenomenology of weak interactions. By considering 
\begin{equation}
\label{W pm}
W^\pm_\mu = \frac{1}{\sqrt{2}}\left(W_\mu^1 \mp i W^2_\mu \right)
\end{equation}
we obtain charged currents which describe the transition from upper and lower components of the weak-isospin doublet observed in nature. Naturally, one would then attempt to identify $W^3_\mu$ with the $Z^0$, however $W^3_\mu$ only couples to left-handed particles or right-handed anti-particles, in contrast to what is observed for the physical $Z^0$. 

\subsection{Electroweak Unification}
A more complete description is thus required to match the theory to the physical reality. The $Z^0$ boson is not the only neutral boson observed in nature: there is also the $\gamma$. We can therefore attempt to include electromagnetic interactions in our description of weak interactions and derive the $Z^0$ and $\gamma$ fields from two neutral fields.

To this aim, we start by introducing \emph{hypercharge}, defined as
\begin{equation}
\label{def hypercharge}
Y = 2(Q-I^{(3)}_W)
\end{equation} 
This is a quantity meant to replace electric charge. It is a quantum number capable of distinguishing the states composing the left-handed doublet from the the one composing the right-handed singlet. The states considered in (\ref{weak doublet}) both have hypercharge $Y=-1$ according to this definition, whereas the $SU_L(2)$ singlet state $\ell^-_R$ has hypercharge $Y=-2$. 

We can now consider the full gauge symmetry for electroweak interactions, $SU_L(2) \otimes U_Y(1)$. Under this new gauge symmetry, the transformations (\ref{weak gauge transformations}) become
\begin{gather}
\label{electroweak gauge transformations}
\psi_L \rightarrow \psi_L^\prime = \exp\left[iy_1\beta(x)\right]\exp\left[i\frac{\tau_j}{2} \alpha^j (x)\right]\psi_L \\
\ell_R \rightarrow \ell_R^\prime = \exp\left[iy_2\beta(x)\right]\ell_R
\end{gather}
where $y_1$ and $y_2$ are the hypercharges of the weak isospin doublet and singlet, respectively. The covariant derivatives thus are
\begin{gather}
\label{EW covariant}
D_\mu \psi_L(x) = \left[\partial_\mu + ig\frac{\tau_j}{2}W^j_\mu(x) + ig^\prime \frac{y_1}{2} B_\mu(x) \right]\psi_L(x) \\
D_\mu \ell_R = \left[\partial_\mu ig^\prime \frac{y_2}{2} B_\mu(x)\right]\ell_R(x)
\end{gather}
where $g$ and $g^\prime$ are the two coupling constants, in general different from one another.

We now have four different gauge bosons, $W^j_\mu(x)$ and $B(x)$, which must be identified with the physical gauge bosons $W^\pm$, $Z^0$ and $\gamma$. The physical W bosons can be identified through the relation in (\ref{W pm}). The mapping from $W^3_\mu$ and $B_\mu$ to $Z_\mu$ and $A_\mu$ can be achieved through a rotation in the neutral sector the gauge bosons
\begin{equation}
\label{mapping}
\begin{bmatrix}
\cos\theta_W & \sin\theta_W \\
-\sin\theta_W & \cos\theta_W
\end{bmatrix}
\begin{bmatrix}
B_\mu \\
W^3_\mu
\end{bmatrix} =
\begin{bmatrix}
A_\mu \\
Z_\mu
\end{bmatrix}.
\end{equation}
If we define the vector
\begin{equation}
\psi = \begin{bmatrix}
\nu_{\ell L} \\
\ell_L\\
\ell_R
\end{bmatrix}
\end{equation} 
and write out the full Lagrangian containing the neutral currents of the electroweak sector of Standard Model,
\begin{equation}
\label{neutral current lagrangian}
\begin{aligned}
\mathcal{L}_{NC} &= \overline{\psi}\gamma_\mu\left\lbrace g \sin\theta_W \frac{\tau_3}{2} + g^\prime \cos\theta_W \frac{Y(\psi)}{2} \right\rbrace \psi A^\mu \\
&+\overline{\psi}\gamma_\mu \left\lbrace g\cos\theta_W \frac{\tau_3}{2}- g^\prime\sin\theta_W\frac{Y(\psi)}{2}\right\rbrace\psi Z^\mu,
\end{aligned}
\end{equation}
we can see that we are required to impose a condition on the coefficients in (\ref{neutral current lagrangian}) to re-obtain the physical currents that we are familiar with. Specifically, the first part of (\ref{neutral current lagrangian}) corresponds to the interaction term of the Lagrangian (\ref{QED lagrangian}) and the second term corresponds to an interaction term involving a second neutral boson. For the sake of simplicity, we can consider the case of an electron for the purpose of this matching.

The interaction term of (\ref{QED lagrangian}), when specifying the left-handed and right-handed components, corresponds to
\begin{equation}
\mathcal{L} = -e\left[ \overline{e}_L\gamma_\mu e_L + \overline{e}_R\gamma_\mu e_R
\right]A^\mu.
\end{equation} 
By inspection, we find that
\begin{equation}
-e = g\sin\theta_W \frac{\tau_3}{2} + g^\prime \cos\theta_W \frac{Y(\psi_{e})}{2}.
\end{equation}
By specifying $\tau_3$ and $Y(\psi_e)$ to the appropriate component of $\psi$, in accordance with Table (1.1), we find that
\begin{equation}
g\sin\theta_W = g^\prime \cos\theta_W = e.
\end{equation}
$\theta_W$ is the \emph{weak mixing angle}, corresponding to the angle of rotation in neutral sector necessary to achieve the desired mapping. Experimentally, $\sin^2\theta_W$ has been measured to be $0.22290 \pm 0.00030$ \cite{NIST}, though theoretically it can be parametrized in terms of other quantities as we shall see in the next section.


\begin{table} 
\begin{center}
\begin{tabular}{ccccc}
\hline 
fermion & $Q$ & $I^{(3)}_W$ & $Y_L$ & $Y_R$ \\ 
\hline 
$\nu_\ell$ & 0 & $+\frac{1}{2}$ & -1 & 0 \\ 
$\ell^-$ & -1 & $-\frac{1}{2}$ & -1 & -2 \\ 
\end{tabular}
\caption{The quantum numbers associated to charged leptons and neutrinos.}
\end{center}
\end{table}

We can also see that the second neutral current in (\ref{neutral current lagrangian}) does indeed correspond to the $Z$ current. It is easy to show that it can be written as
\begin{equation}
\label{NC - Z}
\mathcal{L}_{NC}^{Z} = \overline{\psi}\gamma_\mu \frac{e}{\sin\theta_W \cos\theta_W} Q_Z \psi Z^\mu.
\end{equation}
where $Q_Z = \left\lbrace \frac{\tau_3}{2} - Q\sin^2\theta_W \right\rbrace$. $Q_Z$ is a $3\times 3$ diagonal matrix, allowing for access to each of the components of $\psi$. It needs to be specified in order to find the full coupling constant. This is straightforward for $\nu_{eL}$: the coupling constant turns out to be $\frac{e}{2\sin\theta_W\cos\theta_W}$. 

For the electron, some additional manipulations must first be made since we must deal with the two components. The Lagrangian for the interaction can be written as
\begin{equation}
\mathcal{L}_{NC}^{Ze} = \frac{e}{\sin\theta_W\cos\theta_W}\left\lbrace \overline{e}_L \gamma_\mu Q_Z^L e_L + \overline{e}_R \gamma_\mu Q_Z^R e_R  \right\rbrace Z^\mu.
\end{equation}
The projections can be obtained by considering the operators $P_{L/R} = \frac{1}{2}\left(1 \mp \gamma_5\right)$, i.e.
\begin{equation}
\begin{cases}
\overline{e}_L \gamma_\mu e_L = \overline{e} \gamma_\mu \frac{1}{2}\left(1-\gamma_5\right)e \\
\overline{e}_R \gamma_\mu e_R = \overline{e} \gamma_\mu \frac{1}{2}\left(1+\gamma_5\right)e .
\end{cases}
\end{equation}
The Lagrangian thus becomes
\begin{equation}
\label{NC Ze projection}
\mathcal{L}_{NC}^{Ze} = \frac{e}{\sin\theta_W\cos\theta_W}\left\lbrace \overline{e}\gamma_\mu\frac{1}{2}(Q^L_Z + Q^R_Z)e + \overline{e}\gamma_\mu\frac{1}{2}(Q^L_Z - Q^R_Z)e  \right\rbrace Z^\mu.
\end{equation}
$Q_Z^{L/R}$ can be specified directly from (\ref{NC - Z}) using the values from Table (1.1). We can use those values to specify the couplings which appear in (\ref{NC Ze projection}). The Lagrangian simplifies to
\begin{equation}
\mathcal{L}^{Ze}_{NC} = \frac{e}{2\sin\theta_W\cos\theta_W} \overline{e}\gamma_\mu \left\lbrace v_e - a_e \gamma_5\right\rbrace e Z^\mu
\end{equation} 
where
\begin{equation}
\begin{cases}
v_e = I^{(3)}_W (e_L) \left( 1 + 4Q_e\sin^2\theta_W \right) \\
a_e = I^{(3)}_W (e_L)
\end{cases}
\end{equation}
are the vector and axial components, respectively. 

Thus, we correctly find that the interaction with the Z can involve both $e_L$ and $e_R$ and that the interaction is of the type V-A, with different coupling constants for the two chiral states of the electron. 

The full Lagrangian for electroweak interactions can thus be written as

\begin{equation}
\label{EW lagrangian}
\mathcal{L}_{EW} = \overline{\psi}\left(i\slashed{D} - m\right)\psi - \frac{1}{4}B_{\mu\nu}B^{\mu\nu} - \frac{1}{4}W^j_{\mu\nu}W_j^{\mu\nu},
\end{equation}
which, when combined with the rotation in (\ref{mapping}), gives a correct description of the observed phenomenology.
\section{Spontaneous Symmetry Breaking}

So far, we have only considered Lagrangians which contain massless gauge bosons. This is for a very precise reason: mass terms vary under gauge transformations. If we consider, for example, the Proca action for a generic massive bosonic field in an abelian gauge theory

\begin{equation}
\mathcal{L} = -\frac{1}{4}F_{\mu\nu}F^{\mu\nu} - m^2 A_\mu A^\mu
\end{equation}
it is clear that when we apply the transformation (\ref{gauge field transformation}) the Lagrangian is no longer invariant. This is a significant problem since it is known that the $W^\pm$ and $Z$ bosons are massive.

To solve the problem of massive gauge bosons, it is necessary to introduce the \emph{Brout-Englert-Higgs Mechanism} \cite{PhysRevLett.13.508, PhysRevLett.19.1264}, which induces the spontaneous breaking of the gauge symmetry. 

This mechanism introduces a scalar field, known as the Higgs field, composed of a weak isospin doublet of two complex scalar fields, or equivalently four real scalar fields
\begin{equation}
\phi(x) = 
\begin{bmatrix}
\phi^+ \\
\phi^0
\end{bmatrix} =
\begin{bmatrix}
\frac{1}{\sqrt{2}}(\phi_3 + i\phi_4) \\
\frac{1}{\sqrt{2}}(\phi_1 + i\phi_2)
\end{bmatrix}
\end{equation} governed by a complex $\phi^4$ Lagrangian 
\begin{equation}
\label{SSB lagrangian}
\mathcal{L} = (\partial_\mu\phi)^\dagger(\partial^\mu\phi) - V(\phi, \phi^\dagger)
\end{equation}
where the potential $V(\phi, \phi^\dagger) = \frac{m^2}{2}\phi^\dagger \phi - \frac{\lambda}{4}(\phi^\dagger\phi)^2$. The components of the doublet, as part of the Electroweak sector of the Standard Model, have quantum numbers 
\begin{equation}
\begin{cases}
I_W^{(3)}(\phi^+) = \frac{1}{2} \\
Y(\phi^+) = 1 \\
I_W^{(3)}(\phi^0) = \frac{1}{2} \\
Y(\phi^0) = 1
\end{cases}
\end{equation}
This means that $\phi^+$ carries electrical charge, based on (\ref{def hypercharge}).

The field $\psi_L$ is added to the Lagrangian (\ref{EW lagrangian}), leading to a Lagrangian which remains invariant under a global $SU_L(2)\otimes U_Y(1)$ gauge symmetry, as is easily verifiable. The symmetry can be promoted to a local gauge symmetry, resulting in the following Lagrangian for the Higgs field
\begin{equation}
\mathcal{L} = (D_\mu\phi)^\dagger(D_\mu\phi) - V(\phi,\phi^\dagger)
\end{equation}
where $D_\mu$ is the covariant derivative defined in (\ref{EW covariant}).


By minimizing the potential V, we can identify the ground state of the field
\begin{equation}
\frac{\partial V}{\partial \vert \phi \vert} = m^2\vert \phi \vert + \lambda\vert\phi\vert^3 
\end{equation}
If $m^2 > 0$ and $\lambda > 0$, the minimum occurs when $\phi = 0$. However, if we interpret $m^2$ as a parameter rather than as a mass and allow $m^2 < 0$, we find that there is a local maximum at $\phi = 0$ and a set of minimum value states at 
\begin{equation}
\bra{0}\phi^\dagger \phi\ket{0} \equiv (\phi^\dagger\phi)_0 = -\sqrt{\frac{m^2}{\lambda}}.
\end{equation}

\begin{figure}
\centering
\includegraphics[scale=0.1]{ch1_images/sombrero}
\caption{The ``Mexican hat'' Higgs Potential for an Abelian gauge theory.}
\label{Higgs Potential}
\end{figure}
The quantum vacuum has thus shifted, as represented in Figure \ref{Higgs Potential}. The vacuum is \emph{degenerate} since there are infinite values of $\phi^\dagger\phi$ which minimize $V$. The gauge symmetry is said to be spontaneously broken, the choice of parameters causes the vacuum state and the Lagrangian to no longer share the same symmetry. Since $\lambda$ is an adimensional quantity, it has the dimensions of energy.

Without loss of generality, we can choose for the vacuum states
\begin{equation}
(\phi_1)_0 = -\sqrt{\frac{m^2}{\lambda}} \equiv \frac{v}{\sqrt{2}}, \; (\phi_i)_0 = 0
\end{equation}
where $i = 2,3,4$ and $v$ is the vacuum expectation value of the Higgs field.
Our doublet is thus
\begin{equation}
\phi_0 = \frac{1}{\sqrt{2}}\begin{bmatrix}
0 \\
v
\end{bmatrix}.
\end{equation}

The Lagrangian (\ref{EW lagrangian}) along with (\ref{SSB lagrangian}) remain invariant under the local gauge transformation, however the vacuum state is no longer invariant under neither the $SU_L(2)$ nor the $U_Y(1)$ local gauge symmetries. For example, 
\begin{equation}
\phi_0 \rightarrow \phi_0^\prime = \exp\left[ig \frac{\tau_j}{2}\alpha^j(x)\right]\phi \approx \left\lbrace 1  + ig\frac{\tau_j}{2}\alpha^j(x) + \dots \right\rbrace\phi_0 \neq \phi_0
\end{equation}
Specifically, the invariance is lost due to the action of the generators $\tau_j$. For this reason, these generators are said to be \emph{broken}. Likewise, $Y$ is a broken generator.

Before continuing the discussion, we must first introduce an important theorem involving broken generators.

\begin{theorem}[Goldstone Theorem]
\label{Goldstone Theorem}
For all continuous global symmetries which do not leave the vacuum state unchanged, there exist corresponding massless particles equal in number to the number of broken generators.
\end{theorem}

The theorem holds for global symmetries, however it is also relevant when dealing with local symmetries. In this case, the massless bosons which appear cannot be interpreted as physical particles. They can be gauged away, leading to Higgs-Kibble ghosts which result in \emph{massive} bosons and allow for a physical interpretation of the theory. These ghosts are crucial for our ultimate goal: to give mass to the $W^\pm$ and $Z$ while leaving $\gamma$ massless.

In order to get to our desired result, we must make sure to have one unbroken generator. To obtain it, we can consider the following linear combinations
\begin{equation}
\begin{cases}
Q = \frac{\tau_3}{2} - \frac{Y}{2} \\
Q^\prime = \frac{\tau_3}{2} + \frac{Y}{2}.
\end{cases}
\end{equation}
It is easy to show that $Q$ is an unbroken generator and $Q^\prime$ is broken. In this way we have obtained 3 broken generators ($\tau_1, \tau_2, Q^\prime$) and 1 unbroken ($Q$).

We can now proceed to study the vacuum fluctuations of $\phi$. Naively, these can be written as
\begin{equation}
\phi (x) = \frac{1}{\sqrt{2}}\begin{bmatrix}
\phi_3(x) + i\phi_4(x) \\
v + \phi_1(x) + i\phi_2(x)
\end{bmatrix},
\end{equation}
though, equivalently, we can write
\begin{equation}
\phi (x) = \frac{1}{\sqrt{2}}\exp\left[\frac{iT_j\xi_j(x)}{2} \right]\begin{bmatrix}
0 \\
v + H(x)
\end{bmatrix}
\end{equation}
where $T_j$ are the three $SU(2)$ generators. In the latter form, we have merely parametrized the ``naive'' expression, as can be seen by expanding the exponential term to first order. The fields $\xi_j(x)$ are the ghosts, which we shall gauge away by choosing the unitary gauge
\begin{equation}
\phi (x) \rightarrow \phi^\prime (x) = \exp\left[-\frac{iT_j\xi_j(x)}{2} \right]\phi.
\end{equation}  

In accordance with the gauge protocol, this requires a subsequent modification of $D_\mu$, which leads to a modification of the gauge fields, which are said to ``eat'' the ghosts. After having done so, if we go on to calculate the first term in (\ref{SSB lagrangian}) and use (\ref{W pm}), we find

\begin{equation}
\label{SSB broken}
(D_\mu\phi)^\dagger(D_\mu\phi) = \frac{g^2}{4}\left(v^2 + 2vH + H^2\right)W^+_\mu W^{- \mu} + \frac{1}{8}(v + H)^2 \left( g^2 W^3_\mu W^{3\mu} - 2gg^\prime W^3_\mu B^\mu + g^{\prime 2}B_\mu B^\mu \right)
\end{equation}
which contains all the physically meaningful terms. In particular, we can see that (\ref{SSB broken}) contains Proca mass terms such as 
\begin{equation}
\frac{g^2}{4}v^2 W_\mu^+ W^{-\mu} \equiv M^2_W W_\mu^+ W^{-\mu}.
\end{equation}
We can then follow the same logic as in that used to derive the unified Electroweak theory and map $B_\mu$ and $W^3_\mu$ to $A^\mu$ and $Z_\mu$, respectively. After doing so, the second term in (\ref{SSB broken}) becomes
\begin{equation}
\frac{1}{8}(g^2 + g^{\prime 2})(v + H)^2 Z_\mu Z^\mu
\end{equation}
and we can see that there is no mass term for $A_\mu$, no interaction term involving the Higgs field $H(x)$ and $A_\mu$, and that the Z acquires the mass $M^2_Z = \frac{v^2}{4}(g^2 + g^{\prime 2})$. We can also see that the coupling of the Higgs boson to the other bosons is proportional the bosons' masses. 

We have succeeded in giving mass to the three weak gauge bosons. The choice of parameters $\lambda$ and $m^2$ spontaneously breaks the gauge symmetry, and the interaction of the field $\phi$ with the potential generates would-be Goldstone bosons which manifest as ghosts. The choice of unitary gauge allows for the gauge bosons to eat the ghosts, thus gaining mass. The energy scale at which the gauge symmetry is spontaneously broken, known as the vacuum expectation value, is given by $v$, which in numerical terms corresponds to 246 GeV. Above this energy, the electromagnetic force and the weak nuclear force become one unified electroweak force. 

\section{Yukawa Lagrangian}

A similar problem occurs when considering the mass terms for fermions. In this case, the mass term appearing in the Dirac Lagrangian (\ref{Dirac lagrangian}) does not respect the $SU_L(2)\otimes U_Y(1)$ gauge symmetry due to the fact that the left and right-handed components of the spinor transform differently
\begin{equation}
-m\overline{\psi}\psi = -m\left(\overline{\psi}_R\psi_L + \overline{\psi}_L\psi_R\right).
\end{equation}

This problem can be solved by introducing an interaction with the Higgs field. An infinitesimal $SU(2)$ local gauge transformation has the following effect on the Higgs
\begin{equation}
\phi \rightarrow \phi^\prime = \left\lbrace 1 + ig\frac{\tau_j}{2}\alpha^j(x)\right\rbrace\phi.
\end{equation} 
On the other hand, the same transformation has the opposite effect on $\overline{\psi}_L$
\begin{equation}
\overline{\psi}_L \rightarrow \overline{\psi}^\prime_L = \overline{\psi}_L\left\lbrace 1 - ig\frac{\tau_j}{2}\alpha^j(x) \right\rbrace.
\end{equation}
Therefore, if we consider the combination $\overline{\psi}_L\phi$, we find that this is a gauge invariant quantity. The same holds true for the $U(1)$ gauge symmetry.
Since $\ell_R$ transforms independently from $\psi_L$, we can add it to the combination so as to account for the right-handed component as well. Thus the Lagrangian
\begin{equation}
\label{Yukawa Lagrangian}
\mathcal{L}_Y = -k\left(\overline{\psi}_L\phi\ell_R + \overline{\ell}_R\overline{\phi}\psi_L\right)
\end{equation}
where $k$ is a coupling constant, is invariant under a $SU_L(2)\otimes U_Y(1)$ local gauge transformation.
We can take once again the electron as an example and specify the terms in (\ref{Yukawa Lagrangian}). We find that
\begin{equation}
\mathcal{L}_Y = -\frac{kev}{\sqrt{2}}\left(\overline{e}_L e_R +\overline{e}_R e_L\right) - \frac{keH}{\sqrt{2}}\left(\overline{e}_L e_R +\overline{e}_R e_L\right).
\end{equation}
We thus find the Dirac mass term
\begin{equation}
m_e = \frac{kev}{\sqrt{2}}
\end{equation}
as well as a term which couples the Higgs to the fermion field. This term is proportional to the fermion mass. 

In contrast to the derivation of the gauge bosons' mass, the derivation of the fermionic masses is ad-hoc. The fermionic mass terms depends on the coupling $k$ which must be measured from experiment. There is no explanation for the observed mass hierarchy of the fermions.

It is generally assumed that a different mechanism is responsible for the masses of the neutrinos. Although there is no specific reason as to why neutrinos should not have a Yukawa coupling like the other elementary fermions, given their comparatively small mass, assumed to be of the order of $\sim$meV, they would have to have a coupling orders of magnitude smaller than that of the other fermions. 

Instead, one possible explanation for the mass of the neutrinos is given by the \emph{see-saw mechanism} \cite{zuber}. If the neutrino is a Majorana particle and a sterile, supermassive ($\sim$PeV), right-handed fourth flavor of neutrino exists, it is possible that the large mass of this hypothetical neutrino causes the small mass of the others. Several experiments are currently for evidence which supports this theory.
\section{The Standard Model Lagrangian}

We are now ready to put all the ingredients discussed together and bake the cake that is the Standard Model. The full Lagrangian for the model, written in compact form, is given by
\begin{equation}
\begin{aligned}
\mathcal{L}_{SM} = &-\frac{1}{4}F^{\mu\nu}F_{\mu\nu}\\ 
&+ \bar{\psi}\slashed{D}\psi + h.c.\\
&+ \psi_iy_{ij}\psi_j\phi + h.c. \\
&+ \vert D_\mu\phi \vert^2 - V(\phi).
\end{aligned}
\end{equation}

This Lagrangian describes all possible interactions between all particles, and includes the building blocks described in this Chapter, $\mathcal{L}_{QCD}$, $\mathcal{L}_{EW}$, $\mathcal{L}_{SSB}$ and $\mathcal{L}_Y$ gives mass to the fermions. These interactions are shown in Figure \ref{Standard Model interactions}. 

\begin{figure}
\centering
\includegraphics[width=\linewidth]{ch1_images/sm_interactions}
\caption{They Feynman diagrams of all interactions predicted by the Standard Model \cite{sm_interactions}.}
\label{Standard Model interactions}
\end{figure}

We should mention that, in this form, the Lagrangian is purely classical: it must then be quantized and renormalized \cite{tHooft:1971qjg} in order to fully describe our quantum world. 

	%In due parole, scrivi la lagrangiana per la sola sim SU(2). Parti da lì per unificazione.
	
% To be more precise, the aforementioned symmetry is a vector symmetry since the first-order term of the current obtained is proportional to $\gamma^\mu$.

\nocite{Thomson:2013zua, Ryder:1985wq, Peskin:1995ev, Pich:2012sx}

\end{document}